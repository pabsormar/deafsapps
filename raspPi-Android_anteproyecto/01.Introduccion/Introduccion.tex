\chapter{Introducción}
\label{chap:introduccion}
\onehalfspacing
En los últimos años ha cobrado relevancia y notoriedad el ``Internet de las Cosas'' (en inglés \textit{Internet of Things} o IoT\footnote{Web oficial del IoT \url{http://www.theinternetofthings.eu/}}), que es un concepto que fue por primera vez acuñado por Kevin Ashton en 1999. Este pionero británico impulsó la idea de considerar diferentes dispositivos independientes (``cosas''), de forma que la combinación adecuada de ellos aportara a los usuarios finales un mayor valor añadido y mejores servicios de los que obtendrían por separado.

Paralelamente aunque de forma aislada, las últimas dos décadas han contemplado la irrupción y consolidación de las llamadas tecnologías de movilidad, que han puesto el foco de atención en dispositivos móviles (fundamentalmente teléfonos inteligentes y tabletas) cuyas prestaciones hardware no han parado de crecer y mejorar a lo largo del tiempo. Esta circunstancia, junto con el progresivo auge del ``software libre'', han potenciado un nuevo escenario en el que el usuario está permanente en línea y con pleno acceso a una gran variedad de herramientas y recursos con gran impacto en su día a día.

El presente documento pretende sentar las bases de un proyecto para el desarrollo integral de una aplicación que permita la configuración y gestión remota de un interfono mediante tecnologías de movilidad. El principal objetivo que ha impulsado este trabajo es el de crear una herramienta versátil y sencilla que facilite la interacción del usuario final con este tipo de dispositivos. La aplicación está especialmente dirigida a aquellas personas en situación de movilidad reducida.  

%
\section{Estructura del anteproyecto}
\label{sec:estructura_anteproyecto_deIntro}
Este documento se divide en siete apartados.

Los objetivos y alcances del presente anteproyecto se exponen en el Capítulo \ref{chap:objetivos&Alcances}. El Capítulo \ref{chap:planteamiento} presenta y detalla el problema a resolver, mientras que el Capítulo \ref{chap:justificacion} ilustra y explica la solución adoptada. La metodología de trabajo figura en el Capítulo \ref{chap:metodologia} y, finalmente, los recursos a emplear y el cronograma de trabajo se presentan en los capítulos \ref{chap:recursos} y \ref{chap:cronograma}, respectivamente.

% % %