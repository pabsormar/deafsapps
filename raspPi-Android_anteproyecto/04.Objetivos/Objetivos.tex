\chapter{Objetivos y alcances}
\label{chap:objetivos&Alcances}
\onehalfspacing

\section{Objetivos}
\label{sec:objetivos_deObjetivos&Alcances}
En referencia a los objetivos que se persiguen con el presente trabajo, se distinguen aquellos de carácter finalista de los de índole tecnológico. Los primeros se centran en el origen del problema que se presenta y su resolución, mientras que los segundos versan sobre los desafíos técnicos a abordar durante la implantación de la solución.

Es importante destacar que los llamados objetivos finalistas representan la motivación de este anteproyecto, aunque sin menoscabar la relevancia de los denominados objetivos o desafíos técnicos. Es precisamente la consecución de estos últimos lo que impulsa a un ingeniero a plantearse nuevos retos y proyectos, ya que plantea nuevas posibilidades desconocidas hasta la fecha. 

\subsection{Objetivos finalistas}
\label{subSec:objetivosFinalistas_deObjetivos&Alcances}
Los objetivos que vertebran el presente documento son: 
\begin{itemize}
	\item Promover la utilización de las más recientes tecnologías de movilidad para la mejora de la calidad de vida de personas con limitaciones motoras. 
	%\item Diseñar una aplicación que permita el control remoto de un interfono para un usuario final, que pudiera estar en situación de movilidad reducida.
	\item Analizar y valorar los escenarios y circunstancias en las que la presente iniciativa pueda tener aplicación.
	\item Discutir y consensuar con terceras partes implicadas (usuarios finales, entidades especialistas en el campo, etc.) las finalidades del proyecto a fin de aumentar el número potencial de beneficiarios.
	\item Documentar el proyecto con toda la información requerida para su puesta en funcionamiento, con el fin de obtener una guía rápida de funcionamiento para el usuario final.
\end{itemize}

\subsection{Desafíos técnicos}
\label{subSec:desafiosTecnicos_deObjetivos&Alcances}
Las principales implicaciones técnicas a abordar en el presente proyecto son:
\begin{itemize}
	\item Explorar las posibilidades que ofrece el software libre y las tecnologías de movilidad, para la resolución del problema planteado.
	\item Diseñar e implantar una aplicación móvil que ofrezca al usuario una interfaz gráfica sencilla e intuitiva.
	\item Idear y concretar una topología de conexionado factible que involucre a todos los dispositivos o entidades involucradas.
	\item Sincronizar adecuadamente la comunicación entre un dispositivo móvil y un sistema de control embebido en una red local.		
\end{itemize}

\section{Alcances}
A fin de delimitar el trabajo a acometer en el proyecto, se presentan a continuación los alcances del mismo:
\begin{enumerate}
	\item El proyecto tiene como objetivo el diseño e implantación de una aplicación que facilite el día a día de personas con movilidad reducida.
	\item La solución final aspira a poder ser comercializada, por lo que el trabajo implica una importante labor de documentación. 
	\item El presente trabajo no tiene como finalidad el diseño, fabricación y/o implementación de ningún dispositivo. Todos los elementos hardware a emplear en la solución final adoptada están actualmente en el mercado, y por tanto al alcance de cualquiera particular que lo desee.
	\item El presente trabajo no pretende diseñar o compilar ningún software  o lenguaje de programación nuevo. La solución final adoptada hace uso de diferentes frameworks actualmente en auge, y cuyo funcionamiento conjunto se ha armonizado a fin de producir un resultado satisfactorio y transparente al usuario. 
	\item El grueso de la labor llevada a cabo en este proyecto consiste en hacer que tecnologías nacidas independientemente, o con finalidades distintas, converjan en una única aplicación.
\end{enumerate}

% % %