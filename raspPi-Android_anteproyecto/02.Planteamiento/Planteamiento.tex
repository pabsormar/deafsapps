\chapter{Planteamiento del problema}
\label{chap:planteamiento}
\onehalfspacing

El objetivo principal de este anteproyecto es emplear tecnologías de movilidad para crear un impacto en la vida de personas con dificultades motrices, facilitando tareas aparentemente sencillas para el resto de la población.

Entre los múltiples campos de actuación a explorar, el presente proyecto centra su atención en las tareas domésticas, es decir, aquellas labores que el usuario realiza en su vivienda.

\section{Enunciado y descripción}
\label{sec:enunciado&descripcion_dePlanteamiento}
Supóngase un inmueble propiedad de un individuo con movilidad reducida. Asúmase que la vivienda dispone de una puerta principal y de otra exterior a la calle, cuya apertura puede ser controlada desde un interfono ubicado en el interior del hogar. Es decir, la persona en el inmueble puede utilizar este dispositivo para comunicarse con un eventual visitante, así como permitir su acceso.

\textbf{Enunciado.} El presente proyecto propone el diseño, desarrollo e implantación de un sistema de control remoto para la configuración y gestión del interfono de una vivienda, utilizando para ello tecnologías de movilidad. 
 
\section{Factores y condicionantes}
\label{sec:factore&condicionantes_dePlanteamiento}
Sirva este apartado como esbozo del pliego de condiciones a cumplir en el proyecto.

\begin{enumerate}
	\item El sistema a desarrollar hará uso de frameworks y protocolos englobados dentro del paradigma del software libre, incrementando así el potencial número de beneficiarios.
	\item La solución destacará por su fácil implantación, con apenas molestias para el usuario final.
	\item La aplicación deberá ser extremadamente robusta y segura ante posibles ataques de terceros.
	\item El sistema deberá ser flexible y versátil de forma que pueda adaptarse a potenciales modificaciones de configuración, como por ejemplo la inclusión de varios usuarios.
\end{enumerate}


% % %