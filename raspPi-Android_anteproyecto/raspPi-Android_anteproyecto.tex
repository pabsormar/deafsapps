%-------------------------------------------------------------------------------------
%%% This is an original and self-explanatory Latex document for a PhD thesis. Many comments will follow explaining Latex commands and directives.
%-------------------------------------------------------------------------------------

%% Every file MUST contain '\documentclass[•]{•}', '\begin{document}' and '\end{document}'.
%% The area between '\documentclass[•]{•}' and '\begin{document}' is called the preamble. It normally contains commands that affect the entire document.

% Defining a report with 12pt as main font size, A4 paper format, a new page after the document title, printed on both sides of each page, and chapters starting on right hand pages.
\documentclass[12pt, a4paper, titlepage, oneside, openright]{report}

%\usepackage{titlepic}     % Redefines '\maketitle' to include a picture
\usepackage{fancyhdr}     % Allows adding fancy headers and customize them
%\usepackage{parskip}      % Turns off the paragraph indentation
\usepackage{titlesec}     % Allows to easily customize the headings for the sectional units (Chapters, Sections,...)
%\usepackage{emptypage}    % Removes page numbers from blank ones (created by '\cleardoublepage')
\usepackage{graphicx}     % Allows to use '\includegraphics'
\usepackage{epstopdf}     % Allows to use '.eps' images 
%\usepackage{pdfpages}     % Allows to insert '.pdf' files 
%\usepackage{subfig}       % Allows to define a multiple-panels figure
%\usepackage{wrapfig}      % Allows to wrap text around figures
%\usepackage{multirow}	  % Allows to span cells in several rows within a table
%\usepackage{pbox}         % Allows to break a line inside a table cell
%\usepackage{tablefootnote}     % Allows to use footnotes inside a table
%\usepackage{longtable}    % Allows a table to span multiple pages
%\usepackage{arydshln}    % Allows horizontal and vertical dashed lines in tabulars
\usepackage{url}          % Allows to include an interactive URL link
\usepackage{setspace}     % Allows to set different line spacing formats, such as '\onehalfspacing'
%\usepackage[spanish]{babel}     % Allows content labels in Spanish
%\selectlanguage{spanish}
\usepackage[utf8]{inputenc}     % Allows to write accents directly
%\usepackage{amsmath}      % Allows to include special characters in equations
%\usepackage{amssymb}      % Allows to use 'definition' symbol (a triangle over an equal symbol) thanks to '\triangleq'
%\usepackage{amsfonts}     % Allows to include 'Set of Real numbers' symbol, thanks to '\mathbb{R}'
%\usepackage[titletoc]{appendix}     % Allows to properly include appendices in the outline
%\usepackage[nottoc,notlot,notlof]{tocbibind}     % Allows to include a ToC entry for the bibliography
%\usepackage[breaklinks]{hyperref}     % Allows link text to break across lines (avoids 'overfull \hbox' warning). It also surrounds references and citations with coloured squares. It MUST be loaded as the last package
%\usepackage{breakcites}     % Works with the previous package in order to allow citation breaklines     

% Follows a command definition to address corrections in a different colour (red)
\newcommand{\ammend}{\textcolor{black}}

% Follows a definition to format "\chapter{title}" sections
\titleformat{\chapter}[display]
{\bfseries\LARGE}
{\filright{\chaptertitlename} \Huge\thechapter}
{1ex}
{\titlerule\vspace{1ex}\filleft}
[\vspace{1ex}\titlerule]

\setcounter{tocdepth}{4}    % Determines to which level the sectioning commands are printed in the ToC 
\setcounter{secnumdepth}{4}    % Determines up to what level the sectioning titles are numbered. "4" reaches until 'subsubsection'

%\pagestyle{myheadings}     % Displays the page number on top of the page in the outer corner
\pagestyle{fancy}     % Makes use of the package 'fancyhdr'  
\fancyhead{} \fancyfoot{}     % header and footer from scratch
%\fancyhead[LO, RE]{\slshape \leftmark}     % In case we set up [twoside]   
%\fancyhead[LE, RO]{\slshape \thepage}     % In case we set up [twoside]
\fancyhead[R]{\slshape \thepage}     % In case we set up [oneside]
\fancyhead[L]{\slshape \nouppercase{ \leftmark}}     % In case we set up [oneside]
\setlength{\headheight}{15pt}   % Otherwise, a warning is reported
\renewcommand{\headrulewidth}{0pt}     % No horizontal line in the header

% Starting document
\begin{document}
	
	% The title, date, author, etc. is collectively referred in Latex as 'top matter'
	\title{\textbf{Aplicación para la gestión remota de un interfono mediante tecnologías de movilidad}}
	%\title{\textbf{State-of-the-art Speaker Verification applied to real-time applications}}
	%\title{\textbf{A New GMM System for Text-Independent Speaker Verification under Very Short Duration Conditions}}
	\author{{\small por} \\ {Rafael Boloix Tortosa} \\ {Pablo Luis Sordo Martínez} \\ \\ \\} %{Submitted to Swansea University in fulfilment} \\ {of the requirements for the Degree of} \\ \\ {Doctor of Philosophy} \\ \\ \\ \\ \\ {\textit{\Large Swansea University}}}
	%\date{\today}     % or \date{"Month" "day", "Year"}

	\date{2016}
	%\titlepic{\includegraphics[width=0.45\textwidth]{./0.Figures/swuLogo.pdf}}
	\maketitle   % Without '\maketitle' NO title page is created
	\cleardoublepage
	
	\pagenumbering{roman}
	%\include{Abstract/Abstract}
	%\cleardoublepage
	%\include{Declaration/declaration}
	%\cleardoublepage
	\tableofcontents
	\cleardoublepage
	%\include{Acknowledgement/acknowledgement}
	%\cleardoublepage
	%\include{Dedication/dedication}
	%\cleardoublepage
	\listoffigures
	\cleardoublepage
%	\listoftables
	\cleardoublepage
	%\include{Abbreviations/abbreviations}
	%\cleardoublepage
	%\include{Abbreviations/symbols}
	%\cleardoublepage
	
	\pagenumbering{arabic}
	\chapter{Introducción}
\label{chap:introduccion}
\onehalfspacing
En los últimos años ha cobrado relevancia y notoriedad el ``Internet de las Cosas'' (en inglés \textit{Internet of Things} o IoT\footnote{Web oficial del IoT \url{http://www.theinternetofthings.eu/}}), que es un concepto que fue por primera vez acuñado por Kevin Ashton en 1999. Este pionero británico impulsó la idea de considerar diferentes dispositivos independientes (``cosas''), de forma que la combinación adecuada de ellos aportara a los usuarios finales un mayor valor añadido y mejores servicios de los que obtendrían por separado.

Paralelamente aunque de forma aislada, las últimas dos décadas han contemplado la irrupción y consolidación de las llamadas tecnologías de movilidad, que han puesto el foco de atención en dispositivos móviles (fundamentalmente teléfonos inteligentes y tabletas) cuyas prestaciones hardware no han parado de crecer y mejorar a lo largo del tiempo. Esta circunstancia, junto con el progresivo auge del ``software libre'', han potenciado un nuevo escenario en el que el usuario está permanente en línea y con pleno acceso a una gran variedad de herramientas y recursos con gran impacto en su día a día.

El presente documento pretende sentar las bases de un proyecto para el desarrollo integral de una aplicación que permita la configuración y gestión remota de un interfono mediante tecnologías de movilidad. El principal objetivo que ha impulsado este trabajo es el de crear una herramienta versátil y sencilla que facilite la interacción del usuario final con este tipo de dispositivos. La aplicación está especialmente dirigida a aquellas personas en situación de movilidad reducida.  

%
\section{Estructura del anteproyecto}
\label{sec:estructura_anteproyecto_deIntro}
Este documento se divide en siete apartados.

Los objetivos y alcances del presente anteproyecto se exponen en el Capítulo \ref{chap:objetivos&Alcances}. El Capítulo \ref{chap:planteamiento} presenta y detalla el problema a resolver, mientras que el Capítulo \ref{chap:justificacion} ilustra y explica la solución adoptada. La metodología de trabajo figura en el Capítulo \ref{chap:metodologia} y, finalmente, los recursos a emplear y el cronograma de trabajo se presentan en los capítulos \ref{chap:recursos} y \ref{chap:cronograma}, respectivamente.

% % %
	\chapter{Objetivos y alcances}
\label{chap:objetivos&Alcances}
\onehalfspacing

\section{Objetivos}
\label{sec:objetivos_deObjetivos&Alcances}
En referencia a los objetivos que se persiguen con el presente trabajo, se distinguen aquellos de carácter finalista de los de índole tecnológico. Los primeros se centran en el origen del problema que se presenta y su resolución, mientras que los segundos versan sobre los desafíos técnicos a abordar durante la implantación de la solución.

Es importante destacar que los llamados objetivos finalistas representan la motivación de este anteproyecto, aunque sin menoscabar la relevancia de los denominados objetivos o desafíos técnicos. Es precisamente la consecución de estos últimos lo que impulsa a un ingeniero a plantearse nuevos retos y proyectos, ya que plantea nuevas posibilidades desconocidas hasta la fecha. 

\subsection{Objetivos finalistas}
\label{subSec:objetivosFinalistas_deObjetivos&Alcances}
Los objetivos que vertebran el presente documento son: 
\begin{itemize}
	\item Promover la utilización de las más recientes tecnologías de movilidad para la mejora de la calidad de vida de personas con limitaciones motoras. 
	%\item Diseñar una aplicación que permita el control remoto de un interfono para un usuario final, que pudiera estar en situación de movilidad reducida.
	\item Analizar y valorar los escenarios y circunstancias en las que la presente iniciativa pueda tener aplicación.
	\item Discutir y consensuar con terceras partes implicadas (usuarios finales, entidades especialistas en el campo, etc.) las finalidades del proyecto a fin de aumentar el número potencial de beneficiarios.
	\item Documentar el proyecto con toda la información requerida para su puesta en funcionamiento, con el fin de obtener una guía rápida de funcionamiento para el usuario final.
\end{itemize}

\subsection{Desafíos técnicos}
\label{subSec:desafiosTecnicos_deObjetivos&Alcances}
Las principales implicaciones técnicas a abordar en el presente proyecto son:
\begin{itemize}
	\item Explorar las posibilidades que ofrece el software libre y las tecnologías de movilidad, para la resolución del problema planteado.
	\item Diseñar e implantar una aplicación móvil que ofrezca al usuario una interfaz gráfica sencilla e intuitiva.
	\item Idear y concretar una topología de conexionado factible que involucre a todos los dispositivos o entidades involucradas.
	\item Sincronizar adecuadamente la comunicación entre un dispositivo móvil y un sistema de control embebido en una red local.		
\end{itemize}

\section{Alcances}
A fin de delimitar el trabajo a acometer en el proyecto, se presentan a continuación los alcances del mismo:
\begin{enumerate}
	\item El proyecto tiene como objetivo el diseño e implantación de una aplicación que facilite el día a día de personas con movilidad reducida.
	\item La solución final aspira a poder ser comercializada, por lo que el trabajo implica una importante labor de documentación. 
	\item El presente trabajo no tiene como finalidad el diseño, fabricación y/o implementación de ningún dispositivo. Todos los elementos hardware a emplear en la solución final adoptada están actualmente en el mercado, y por tanto al alcance de cualquiera particular que lo desee.
	\item El presente trabajo no pretende diseñar o compilar ningún software  o lenguaje de programación nuevo. La solución final adoptada hace uso de diferentes frameworks actualmente en auge, y cuyo funcionamiento conjunto se ha armonizado a fin de producir un resultado satisfactorio y transparente al usuario. 
	\item El grueso de la labor llevada a cabo en este proyecto consiste en hacer que tecnologías nacidas independientemente, o con finalidades distintas, converjan en una única aplicación.
\end{enumerate}

% % %
	\chapter{Planteamiento del problema}
\label{chap:planteamiento}
\onehalfspacing

El objetivo principal de este anteproyecto es emplear tecnologías de movilidad para crear un impacto en la vida de personas con dificultades motrices, facilitando tareas aparentemente sencillas para el resto de la población.

Entre los múltiples campos de actuación a explorar, el presente proyecto centra su atención en las tareas domésticas, es decir, aquellas labores que el usuario realiza en su vivienda.

\section{Enunciado y descripción}
\label{sec:enunciado&descripcion_dePlanteamiento}
Supóngase un inmueble propiedad de un individuo con movilidad reducida. Asúmase que la vivienda dispone de una puerta principal y de otra exterior a la calle, cuya apertura puede ser controlada desde un interfono ubicado en el interior del hogar. Es decir, la persona en el inmueble puede utilizar este dispositivo para comunicarse con un eventual visitante, así como permitir su acceso.

\textbf{Enunciado.} El presente proyecto propone el diseño, desarrollo e implantación de un sistema de control remoto para la configuración y gestión del interfono de una vivienda, utilizando para ello tecnologías de movilidad. 
 
\section{Factores y condicionantes}
\label{sec:factore&condicionantes_dePlanteamiento}
Sirva este apartado como esbozo del pliego de condiciones a cumplir en el proyecto.

\begin{enumerate}
	\item El sistema a desarrollar hará uso de frameworks y protocolos englobados dentro del paradigma del software libre, incrementando así el potencial número de beneficiarios.
	\item La solución destacará por su fácil implantación, con apenas molestias para el usuario final.
	\item La aplicación deberá ser extremadamente robusta y segura ante posibles ataques de terceros.
	\item El sistema deberá ser flexible y versátil de forma que pueda adaptarse a potenciales modificaciones de configuración, como por ejemplo la inclusión de varios usuarios.
\end{enumerate}


% % %
	\chapter{Justificación de la solución adoptada}
\label{chap:justificacion}
\onehalfspacing

Este capítulo ilustra y describe la solución adoptada para la problemática planteada en el presente anteproyecto.

La Figura \ref{fig:solucionTopologia} muestra un diagrama en el qu ese presentan las distintas entidades que componen la topología. 
\begin{figure}[h]
	\centering
	\includegraphics[width=\linewidth]{./00.Figures/solucionTopologia.eps}
	\caption{\textit{Esquema de la solución adoptada.}}
	\label{fig:solucionTopologia}
\end{figure}

\begin{enumerate}
	\item Interfono exterior.
	\item Auricular en el interior de la vivienda conectado a un dispositivo Raspberry Pi 2 B. 
	\item Dispositivo móvil Android.
\end{enumerate}

El sistema requiere de un entorno WiFi, como el disponible habitualmente en cualquier vivienda. 

El funcionamiento es el siguiente:
\begin{itemize}
	\item Un visitante acude a la vivienda y acciona el interfono exterior.
	\item En el interior del inmueble el auricular suena con normalidad, indicando además el evento a la Raspberry Pi.
	\item El teléfono móvil del usuario, en el interior de la vivienda, suena como si de una llamada se tratase.
	\item Ambos dispositivos establecen un enlace directo gracias al framework de Android WiFi P2P\footnote{\url{https://developer.android.com/guide/topics/connectivity/wifip2p.html}}, permitiendo la comunicación entre ambos individuos.
	\item Finalmente el inquilino, si así lo desea, abre la puerta de la vivienda desde su dispositivo móvil. La señal es recibida por la Raspberry Pi, que hace las veces de actuador. 
\end{itemize}

% % %
	\chapter{Metodología de trabajo}
\label{chap:metodologia}
\onehalfspacing

Las tareas a realizar en el presente proyecto se organizarán en módulos de trabajo independientes. De esta forma, los distintos resultados intermedios alcanzados podrán integrarse fácilmente en la solución final.

%La Figura 5.1 ilustra las labores a realizar para la consecución del presente proyecto.

% % %
	\chapter{Recursos a emplear}
\label{chap:recursos}
\onehalfspacing

\section{Frameworks involucrados}
\label{sec:frameworksInvolucrados_deRecursos}
\begin{itemize}
	\item Android, basado en Java SE. La comunicación entre los dispositivos se realizará mediante la API WiFi P2P.
	\item Unix. Se usa una distribución de Raspbian, un sistema operativo basado en Unix específicamente desarrollado para los dispositivos Raspberry.
\end{itemize}

\section{Recursos hardware}
\label{sec:recursosHardware_deRecursos}
\begin{itemize}
	\item Interfono estándar de una vivienda. La idea es que en futuras fases del proyecto se emplee un dispositivo que incorpore una cámara de vídeo.
	\item Raspberry Pi 2 B.
	\item Terminal Android, concretamente un teléfono con la versión 4.4.1 del sistema operativo.
\end{itemize}

% % %
	\chapter{Cronograma}
\label{chap:cronograma}
\onehalfspacing

% % %
	%
	% The IEEEbib.bst bibliography style file from IEEE produces unsorted bibliography list
	% The 'apalike' style allows using a Harvard-format citing style, i.e [author, year]
	\bibliographystyle{apalike}
	%\bibliography{../allPapers}
	
	%\begin{appendices}
	%	\include{Appendices/asvFoundations}
	%	\include{Appendices/biometricVocabulary}
	%	\include{Appendices/databaseCharacteristics}
	%	\include{Appendices/theEmAlgorithm}
	%	\include{Appendices/Publications}
	%\end{appendices}
	%\includepdf[pages=-]{Publications/I4U13_IS-1}
	%\includepdf[pages=-]{Publications/Sordo14_IWBF-1}
	%\includepdf[pages=-]{Publications/Larcher14_IS-1}
	
\end{document}
%% The reason for marking off the end of your text is to provide a place for LaTeX to be programmed to do extra stuff automatically at the end of the document, like making an index.
% % %

%Introducción
%Planteamiento del problema
%Justificación
%Objetivos
%Antecdentes
%Hipótesis
%Metodología
%Recursos materiales y humanos
%Alcances o metas
%Cronograma
%Citas y referencias bibliográficas